Snow cover variation is a sensitive indicator of environmental change. Therefore, snow cover mapping is a critical exercise which is supposed to be performed at regular intervals. Observations at periodic intervals require a sensor independent of time and weather. Microwave remote sensing is a reliable tool for the timely observation of spatial and temporal variability of snow-pack characteristics. 

Approaches like~\cite{slater1999potential} proposed multi-sensor constellations for quantifying snow-cover, especially in the sub-polar maritime regions. They highlighted the problems faced in mapping the ephemeral nature of snow in such regions, in light of lack of access to imagery on occasion due to cloud cover. Although, a multi-sensor approach can be sufficient for snow-mapping, it is usually economically infeasible to cover continent scale areas, if the data is to be purchased. Microwave based approaches, which are unperturbed by the presence of cloud cover, are more practical and efficient for the task. In~\cite{grody1991classification} a seven-channel microwave radiometer was demonstrated for snow-cover mapping. Such a passive sensor is able to operate continuously, and with its large footprint, is able to cover large areas of the Earth's surface. At the same time, due to the nature of microwave radiation it is not affected by cloud cover. However, the disadvantage is that the resolution is coarse. Active SAR sensors have a much higher spatial resolution, while also being unaffected by cloud cover~\cite{shi1994snow}. Although active sensors have a smaller coverage than passive ones, they have far increased ability to mapped details. SAR is a side-looking imaging technique, which leads to geometric distortions like layover and shadow. These artifacts can be mitigated by using a hybrid approaches that merges optical and SAR data~\cite{koskinen1999snow}.
%
Advanced SAR techniques like InSAR have been explored to improve classification performance, especially in challenging conditions like wet snow. In~\cite{strozzi1999mapping} the degree of coherence was used to delineate wet-snow, in areas where backscatter alone was insufficient. PolSAR techniques can also be used to recover information in such conditions. In~\cite{surendar2013improved} an advanced technique to estimate snow wetness has been described, which can be extended to improve the accuracy of snow-cover classification maps.

The capabilities assessment of several polarimetric parameters including a supervised classification technique to discriminate snow-covered areas was carried out over the Indian Himalayan region~\cite{singh2014capability}. The proposed polarization fraction ($PF$) scheme for wet snow mapping, in the absence of any training samples, performed better than the rest of the classification techniques. Although quad-pol ALOS-PALSAR L-band SAR data demonstrated significant improvement, however, an analysis of higher frequencies (S- or C-band) was much desired~\cite{rott1987possibilities}. 

Snow-covered areas were segregated from the snow-free/forested areas through optimal values of transmitted and received polarization states. Since the seasonal polarimetric contrast optimization was noted to be limited by significant variability in the polarimetric response of the underlying surface, the PCVE approach was proposed~\cite{martini2006dry}. An adaptive wet snow mapping algorithm using ALOS-2 L-band full-polarimetric image was proposed with a supervised learning approach using quaternion neural networks (QNN)~\cite{usami2016polsar}.
%
Muhuri et al. proposed two most recent algorithms for snow cover mapping over the India Himalayas using the Radarsat-2 polarimetric SAR data~\cite{Muhuri_Scattering_Mechanism_2017,Muhuri_Seasonal_Snow_Cover_2017}.

In~\cite{Muhuri_Scattering_Mechanism_2017}, a snow cover mapping algorithm has been proposed by analyzing the variation in the target scattering decomposition parameters~\cite{touzi2007target, lee2009polarimetric} in the winter-summer image pairs. In~\cite{Muhuri_Seasonal_Snow_Cover_2017}, a computationally simple yet an effective snow extent mapping method is proposed by exploiting the ratio of the seasonal variation of co-polarized $(hh\mbox{-}vv)$ correlation coefficient and the total scattering power. The difference image is obtained by temporal (winter-summer) ratioing of this index. 

%In this work, a snow cover mapping framework is proposed which uses the Poincar\'e sphere parameters obtained from the Radarsat-2 full-polarimetric SAR images. An unsupervised Auto-Encoder (AE) is used to learn an optimal representation of these parameters. This is followed by a Feed Forward (FF) network which generates the final snow-cover map as a two-class classification process.